\documentclass{omgrpt}
\usepackage{lipsum}

\begin{document}
\oclHeadingOne{Introduction}\label{sec:ocl-1:introduction}
The \LaTeX{} style \texttt{omg.sty} tries to mimic the layout used by
the OMG for the UML/OCL standard documents. To achieve this, the style
\begin{itemize}
\item sets the paper size to 8.28x11 inch
\item sets the text size to 6.7x8 inch
\item uses the PostScript fonts Times Roman, Helvetica, and Courier
\item formats paragraphs without initial indentation and ragged 
  right formatting
\end{itemize}
Note that this style works only in combination with one of the
KomaScript classes, e.g., \texttt{scrbook} (replaces \texttt{book}) or 
\texttt{scrreprt} (replaces \texttt{report}). As shorthands, this
package also provides the \texttt{omgrpt} class.

Moreover,  \texttt{omg.sty} provides the following \LaTeX-commands:
\begin{itemize}
\item New commands for chapters/sections:
\begin{itemize}
\item \verb|\oclHeadingOne{Examples}| for top-level sections
    (chapters), e.g., \autoref{ocl-1:examples} or ``9 Concrete
    Syntax''% in~\cite{omg:ocl:2014}.
\item \verb|\oclHeadingTwo{Examples}| for second-level sections,
    e.g., \autoref{ocl-2:examples}  or ``9.1 Structure of the Concrete
    Syntax''% in~\cite{omg:ocl:2014}.
\item \verb|\oclHeadingThree{Examples}| for third-level sections,
    e.g., \autoref{ocl-3:examples} or ``9.2.1 Parsing''% in~\cite{omg:ocl:2014}.
\item \verb|\oclHeadingFour{Examples}| for fourth-level sections,
    e.g., \autoref{ocl-4:examples} (should be used carefully).
\item \verb|\oclHeadingZero{Examples}| for fourth-level sections
    without numbering,
    e.g., \autoref{ocl-0:examples} or ``Abstract syntax mapping'' (in
    ``9.4.1 ExpressionInOclCS'')% in~\cite{omg:ocl:2014}).
  \end{itemize}
\item New environments
% ocl-definition e.g. "ExpressionInOclCS.ast : OclExpression" as in 9.3.1
  \begin{itemize}
  \item  \verb| \begin{oclDefintion} \ldots  \end{oclDefinition}| for
    OCL definitions, e.g., ``ExpressionInOclCS.ast : OclExpression''
    as in 9.3.1% in~\cite{omg:ocl:2014}.
  \begin{oclDefinition}
    ExpressionInOclCS.ast : OclExpression
  \end{oclDefinition}
  \end{itemize}
\item New styles for emphasizing or highlighting texts
  \begin{itemize}
  \item  \verb|\oclEmph{declaration}| (italics) for declarations and segments of
OCL text, e.g., \oclEmph{declaration}.
  \end{itemize}
\end{itemize}

\oclHeadingOne{Examples}\label{ocl-1:examples}
\lipsum[1]

\begin{figure}
  \centering
  \centerline{\Large\textbf{Example Figure}}
  \caption{This is an example figure caption}
\end{figure}

\begin{table}
  \centering
  \centerline{\Large\textbf{Example Table}}  
  \caption{This is an example table caption}
  \label{tab:asdf}
\end{table}

\oclHeadingTwo{Lore Ipsum}\label{ocl-2:examples}
\lipsum[2-3]

\oclHeadingThree{Lore Ipsum}\label{ocl-3:examples}
\lipsum[2-3]

\oclHeadingFour{Lore Ipsum}\label{ocl-4:examples}
\lipsum[2-3]

\oclHeadingZero{Lore Ipsum}\label{ocl-0:examples}
\lipsum[2-3]

\end{document}



\end{document}


